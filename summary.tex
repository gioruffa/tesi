\pdfminorversion 4
\documentclass[a4paper,10pt]{article}
\usepackage[english]{babel} 
\usepackage{fixltx2e}

% Dimensione della pagina
\setlength{\oddsidemargin}{.1in}  % Distance from the left edge -1 inch 
\setlength{\textwidth}{165mm}     % Normal width of the text
\setlength{\topmargin}{.25in}     % Distance from top to PAGE'S HEAD -1 inch
\setlength{\textheight}{225mm}    % Height of the body of page
\setlength{\headheight}{0mm}      % Height of a box containing the head
\setlength{\parskip}{0.5mm}         % Extra vertical space before a paragraph
\setlength{\parindent}{4mm}       % Width of the indentation 
\linespread{1.12}                 % Line spacing

\begin{document}

\thispagestyle{plain}
\begin{center}
	\vspace*{-4cm}
    \Large
    \textbf{ Tuning the computational architecture for Quantum Espresso\\ab initio calculation of nanostructures }
    
    \vspace{0.3cm}
    \textbf{Giorgio Ruffa}
    
	\begin{small}
	\textit{Advisor:} Dott. Dario Tamascelli ~~ \textit{Co-Advisor:} Prof. Michele Ceotto ~~ \textit{External Advisor:} Dott. Davide Ceresoli
	\end{small}
	
    
    \vspace{0.2cm}
    \textbf{Summary}
\end{center}

Quantum Espresso (QE) is one of the most reliable and widespread  suite of codes for quantum chemistry calculations of nanostructures\cite{QE}. 
Given the relevance of QE for the condensed matter physics community, the aim of this work was to find tuning strategies to optimize QE performances on two different computational architectures: 
the CINECA Galileo multicomputer cluster and the Sgi UV2000 shared-memory CC-NUMA multiprocessor.

To fully understand the behavior of QE, a considerable amount of preparatory work was performed.
First we studied the theoretical foundations of \textit{ab initio} calculations, familiarizing with the fundamental Hartree-Fock (HF) self consistent field (SCF) method\cite{Attila}.
Then, starting from the computational limits of HF method, we recalled density functional theory (DFT)\cite{Parr} upon which QE is based.
We highlighted how, in this framework, the variational approach brings a new, computationally efficient, SCF formulation\cite{Martin}.

Only at this point we were able to fully understand how each step of the SCF cycle has been implemented\cite{Marx}.
Therefore, we highlighted which algorithm are used to perform the calculations, which are the subroutines implementing each step of the SCF cycle and which are the computational complexities of each relevant subroutine.


Once the ``mechanics" of QE were clear, we underlined all the differences between the two computational architectures, which mainly lies in the presence or absence of global shared memory and the interconnect technology adopted\cite{Tanenbaum}.
We also performed a series of introductory tests to find the main factors that could bias our measurements and we reported precise methods to minimize them. 
We also outlined all the critical aspects that rise when trying to evaluate the performance of an SCF algorithm; namely the numerical instability associated both to the number of iterations needed to reach self consistency and the differences in the execution time of each iterative step.
An in-depth analysis of communication (MPI) patterns within the code was also performed.
The theoretical and computational knowledge we gained was essential to give us the necessary tools to obtain solid results. 
These results can be divided in two main categories: architecture independent tuning and architecture dependent tuning.

For architecture independent tuning we discovered that, in contrast to what reported in literature, multi-threading parallelization (OpenMP) can have a strongly positive impact when a relatively low number of cores is employed.
This result can be of great interest for hight throughput computing scenarios or when shared computational resources are overloaded and low parallelized runs performed locally can lead to faster results than high parallelized jobs waiting to be executed on massive, nevertheless overcrowded, clusters.

With an architecture dependent perspective we found that the UV2000 multiprocessor is more suited to perform calculations of ``\textit{medium size}" systems\cite{Titania1}\cite{prace} (one hundred to two hundreds atoms in surface configuration) whereas Galileo cluster has a general advantage on ``\textit{small size}" systems (approximately fifty atoms).
We have also performed a brief test with a ``\textit{large}" organic OLED crystal confirming the advantage of the UV2000.

We also found that UV2000's performances overtakes Galileo's ones when two or more Galileo's cluster nodes are employed, suggesting a more efficient handling of the interprocess communication. 
This hypothesis was confirmed by the profiling metrics obtained with professional tools and by comparing the behavior of QE's communication subroutines.

We also found that linear algebra constitutes a major bottleneck above the eight cores landmark and the methods to optimize its performance are very different between the two architectures.
The best results from Galileo cluster are achieved reducing the number of cores dedicated to linear algebra (as suggested in litterature \cite{QE2}), whereas on the UV2000 multiprocessor the opposite strategy leads to the best results, marking a clear advantage for shared-memory architectures.

Applying the resulting tuning strategies, we were able to sensibly boost the performance of the code and finally reach a linear scaling on both the architectures.

\begin{thebibliography}{9}


\bibitem{QE}
P. Giannozzi , S. Baroni , N. Bonini,``QUANTUM ESPRESSO: a modular and open-source software project for quantum simulations of materials", J. Phys.: Condens. Matter \textbf{21} 395502 (19pp), (2009).

\bibitem{Attila}
A. Szabo, N. S. Ostlund,
Modern Quantum Chemistry: Introduction to Advanced Electronic Structure Theory,
Dover Pubblications, New York,
1996.


\bibitem{Parr}
R. G. Parr, W. Yang,
Density Functional Theory of Atoms and Molecules,
Oxford University Press,
1989.


\bibitem{Martin}
R. Martin, 
Electronic Structure,
Cambridge University Press,
2008.


\bibitem{Marx}
D. Marx, J. Hutter,
Ab initio Molecular Dynamics,
Basic Theory and Advanced Methods,
Cambridge University Press,
2009.


\bibitem{Tanenbaum}
A. S. Tanenbaum, T. Austin,
Structured Computer Organization,
Sixth edition,
Pearson,
2012


\bibitem{Titania1}
C. Marchiori, G. Di Liberto, G. Soliveri, L. Loconte, L. Lo Presti, D. Meroni, M. Ceotto, C. Oliva, S. Cappelli, G. Cappelletti, C. Aieta, S. Ardizzone,
``Unraveling the Cooperative Mechanism of Visible-Light Absorption in Bulk N,Nb Codoped TiO\textsubscript{2} Powders of Nanomaterials",
The Journal of Physical Chemistry C \textbf{118} 41, (2014).

\bibitem{prace}
I. Girotto, N. Varini, F. Spiga, C. Cavazzoni, D. Ceresoli, L. Martin-Samos, T. Gorni,
``Enabling of Quantum ESPRESSO to Petascale Scientific Challenges",
PRACE Whitepapers,
PRACE,
(2012).

\bibitem{QE2}
P. Giannozzi, C. Cavazzoni,
``Large-scale computing with QUANTUM Espresso",
Il Nuovo Cimento, Vol. \textbf{32 C}, N. 2, (2009).






\end{thebibliography}

\end{document}