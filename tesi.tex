% draft 3rd year diploma thesis, by Nick Manini, 2013/03/22
\documentclass[a4paper,12pt]{article}
\usepackage[english]{babel} % or other languages, e.g:
%\usepackage[italian]{babel} % needs debian package texlive-lang-italian
%\usepackage[latin1]{inputenc} % use to reproduce accented characters correctly
\usepackage{hyperref}
\usepackage{graphicx}
\usepackage{amsmath}
\usepackage{amssymb}
\usepackage{physics}
\usepackage[width=125mm]{caption}

\usepackage[document]{ragged2e}

% Dimensione della pagina
\setlength{\oddsidemargin}{.3in}  % Distance from the left edge -1 inch 
\setlength{\textwidth}{145mm}     % Normal width of the text
\setlength{\topmargin}{.25in}     % Distance from top to PAGE'S HEAD -1 inch
\setlength{\textheight}{225mm}    % Height of the body of page
\setlength{\headheight}{0mm}      % Height of a box containing the head
\setlength{\parskip}{0.5mm}         % Extra vertical space before a paragraph
\setlength{\parindent}{9mm}       % Width of the indentation 
\linespread{1.12}                 % Line spacing        
\renewcommand{\floatpagefraction}{.9}

\usepackage{xcolor}
\newcommand\mynotes[1]{\begin{flushright}\textcolor{red}{TODO: #1}\end{flushright}}

\begin{document}

\title{\bf \Huge Titolazzo della tesi, if seems long\\go to newline }


\author{Giorgio Ruffa\\
Dipartimento di Fisica, Universit\`a degli Studi di Milano,\\
Via Celoria 16, 20133 Milano, Italia
}
\date{April 22, 2016} % the exact date of graduation, when available


{
\thispagestyle{empty}

\centerline{
\includegraphics[width=120mm,angle=0,clip=]{UniversitasMediolanensis.eps}
}

\begin{center}
{\Large Facolt\`a di Scienze e Tecnologie\\
\vskip0.2cm Laurea Triennale in Fisica }
\end{center}


\vskip1.5cm
\begin{center}
{\huge \textbf{Tuning the computational architecture for Quantum Espresso ab initio calculation of nanostructures}}
\end{center}

{\large
\vskip20mm Relatore:  Dott. Dario Tamascelli
\vskip 1mm Correlatore: Prof. Michele Ceotto\\
\vskip 1mm Relatore Esterno: Dott. Davide Ceresoli\\
}

\vskip2cm
\hskip9cm\parbox[t]{7cm}
{\large 
Giorgio Ruffa\\
Matricola n$^\circ$ $742031$\\
A.A. $2014$/$2015$\\
\vskip 0.5mm Codice PACS: 31.15.-p
}

\newpage
\newpage
\thispagestyle{empty}
\clearpage
}

 % eccezionalmente qui include il frontespizio
% in general avoid \include altoghether, it is looking for trouble !!

\newpage\qquad
\newpage

\maketitle

%---------------------------------------------------------
\begin{abstract}

A long time ago in a galaxy far, far away...

\vskip0.75cm
\hskip5cm
\parbox[t]{7cm}
{
Advisor: {\it Prof. Dario Tamascelli}\\
Co-Advisor: {\it Prof. Michele Ceotto}
}
\end{abstract}
%--------------------------------------------------------

\newpage
\tableofcontents
\newpage

%----------------------------------------------------------------------------
\section{Quantum Espresso}
%----------------------------------------------------------------------------

Quantum ESPRESSO is an integrated suite of Open-Source computer codes for electronic-structure calculations and materials modeling at the nanoscale.
It is based on density-functional theory, plane waves, and pseudopotentials.

\mynotes{Scritto per girare su un gazilione di architetture}

\section{Theoretical Introduction}\label{model:sec}

This section will cover the basic theory upon which Quantum ESPRESSO is based.

By starting with a very general approach we can say that the Hamiltonian associated to a system of atoms with $N_N$ nuclei and $N_e$ electrons can be written as:


\begin{equation}\label{eq_theHamiltonian}
\hat{H}_{tot} = \hat{T}_{N} + \hat{T}_{e} + \hat{V}_{Ne} + \hat{V}_{NN} + \hat{V}_{ee}
\end{equation}

Where:

\begin{equation}
\hat{T}_{N} = \frac{\hbar}{2} \sum_{\alpha}^{N_N} \frac{\nabla_{\alpha}^2}{M_{\alpha}}
\end{equation}
Is the kinetic energy of the nuclei

\begin{equation}
\hat{T}_{e} = \frac{\hbar}{2m_{e}} \sum_{i}^{N_e} \nabla_{i}^2
\end{equation}
Is the kinetic energy of the electrons

\begin{equation}
\hat{V}_{Ne} = -\frac{e^2}{4\pi\epsilon_{0}} \sum_{i}^{N_e}\sum_{\alpha}^{N_N} \frac{Z_{\alpha}}{\mid R_{\alpha} - r_{i} \mid }
\end{equation}
is the electron-ion attraction potential energy

\begin{equation}
\hat{V}_{NN} = \frac{e^2}{4\pi\epsilon_{0}} \frac{1}{2} \sum_{\alpha \neq \beta}^{N_N} \frac{Z_{\alpha} Z_{\beta}}{\mid R_{\alpha} - R_{\beta} \mid }
\end{equation}
is the nucleus-nucleus repulsion potential energy

\begin{equation}
\hat{V}_{ee} = \frac{e^2}{4\pi\epsilon_{0}} \frac{1}{2} \sum_{i \neq j}^{N_e} \frac{1}{\mid r_{i} - r_{j} \mid }
\end{equation}
is the electron-electron repulsion potential energy

A state function $\ket{\psi}$ describing all the particles involved in the system will evolve following the Schrodinger equation.
\begin{equation}\label{eq:eq_sch}
	i\hbar\dv{t}\ket{\psi(t)} = \hat{H}_{tot}\ket{\psi(t)}
\end{equation}

Although the universality of this equation, we know very well that even a simple molecule like $H_2^{+}$ has no analytical solution.

Thus, even from a computational standpoint, a set of approximations must be performed.


\subsection{The Bohr-Oppenheimer Approximation}

The Bohr-Oppenheimer approximation takes note of the great difference in masses of electrons and nuclei.
Nuclear mass is much higher than electron mass, so we expect that the electrons will have much higher velocities than the nuclei. 

It's now reasonable to separate the motion of the system in two distinct movements: the "slow" movement of the nuclei, and the "fast" movement of the electrons.

We can say that from the point of view of electrons, the nuclei appear to be fixed. 
From a physical standpoint the electrons are moving in the static field produced by the nuclei while still interacting within each others \cite[p.241]{Atkins97}.


Using this approximation the kinetic energy of the nuclei $\hat{T}_{NN}$ can be neglected and the repulsion between the nuclei $\hat{V}_{NN}$, can be considered to be constant.

We will consider now the electronic Hamiltonian $H_{elec}$ :
\begin{equation}
	\hat{H}_{elec} = \hat{T}_{e} + \hat{V}_{Ne} + \hat{V}_{ee}
\end{equation}

Rewriting $H_{elec}$  using atomic units:
\begin{equation}\label{eq:H_elec}
	H_{elec} = - \sum_{i}^{N_{e}} \frac{1}{2} \nabla_{i}^2  - \sum_{i}^{N_{e}} \sum_{\alpha}^{N_{\alpha}} \frac{Z_{\alpha}}{r_{i\alpha}}  + \frac{1}{2} \sum_{i \neq j}^{N_{e}} \frac{1}{r_{ij}}
\end{equation}

The associate Schroedinger equation is,
\begin{equation}
	H_{elec} \Phi_{elec} = \epsilon_{elec} \Phi_{elec}
\end{equation}

With solution
\begin{equation}
	\Phi_{elec} = \Phi_{elec}(r_{\{i\}};R_{\{\alpha\}})
\end{equation}

Where the dependency from the electronic coordinates $\{r_i\}$ is explicit, but the dependency from nuclear coordinates $\{R_{\alpha}\}$ is parametrical.
This implies that also the electronic energy depends parametrically on $\{R_{\alpha}\}$

\begin{equation}
	\epsilon_{elec} = \epsilon_{elec}(\{R_{\alpha}\})
\end{equation}

To obtain the total energy (with fixed nuclei) we have to add the constant ion-ion Coulomb potential energy

\begin{equation}\label{eq:totEn1}
	\epsilon_{tot} = \epsilon_{elec} + \frac{1}{2} \sum_{\alpha \neq \beta}^{N_N} \frac{Z_{\alpha} Z_{\beta} }{R_{\alpha \beta}}
\end{equation}

Equation from \eqref{eq:H_elec} to \eqref{eq:totEn1} constitutes the so called \textit{``Electronic Problem"}.

Once solved one could then apply the same principle to the nuclear problem.
Since the electrons moves much faster then the nuclei, it is reasonable to approximate \eqref{eq_theHamiltonian} by replacing electronic coordinates by their average value, averaged over $\Phi_{elec}$.

The nuclear soluzion $\Phi_{nucl}(\{R_{\alpha}\})$ will describe vibration, rotation, and translation of the molecule.

The complete approximate solution to \eqref{eq_theHamiltonian} will be  \cite[p.43-45]{Attila}

\begin{equation}
	\Phi(\{r_i\};\{R_{\alpha}\}) = \Phi_{elec}(\{r_i\};\{R_{\alpha}\})~\Phi_{nucl}(\{R_{\alpha}\})
\end{equation}



\subsection{The electron-electon Potential Problem}

The Hamiltonian \eqref{eq:H_elec} can be rewritten in the following form:








\section{Computational Architectures} \label{comparch:sec}


We have tested on NUMA and GALILEO and ...


\subsection{NetXScale Cluster Architecture}\label{galileoarch:sec}
\subsection{NUMA-CC Architecture}\label{numaarch:sec}


\section{Discussion and Conclusion}

May the force be with you.

%In this work, we relate the insect concentration fluctuations to the
%dissipation of Milan University budget to catering companies.
%%
%We find it appropriate to name the present result ``{\em
%fluctuation-dissipation theorem}''.


%------------------------------------------------------------------
%  BIBLIOGRAPHY
%------------------------------------------------------------------
\clearpage
\addcontentsline{toc}{section}{Bibliography}
\begin{thebibliography}{9}

% note that the references must be listed in the same order as they are cited
% in the text above:

%\bibitem{Manini82} % standard format for a book:
%X. Manini and G. D'Annunzio,
%{\it Terra Vergine} (Springer-Verlag, Berlin, 1882).
%
%\bibitem{Jones55} % standard format for a journal article:
%M. Jones and J. Mones, J. Irreprod. Results {\bf 83}, 2322 (1955).
%
%\bibitem{Dragoni11}  % standard format for a thesis:
%D. Dragoni, {\it Interfacial layering of ionic liquids on solid surfaces},
%diploma thesis (University Milan, 2011),
%\url{http://www.mi.infm.it/manini/theses/dragoni.pdf}.

\bibitem{Atkins97}
P. W. Atkins and R. S. Friedman,
Molecular Quantum Mechanics,
Oxford University Press, New York,
3rd Edition,
1997.

\bibitem{Attila}
A. Szabo, N. S. Ostlund
Modern Quantum Chemistry: Introduction to Advanced Electronic Structure Theory,
Dover Pubblications, New York,
1996.
\end{thebibliography}
\end{document}
